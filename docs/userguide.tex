\documentclass[usenatbib,11pt]{article}
\pagestyle{myheadings}
\markright{OBERON - OBliquity and Energy balance Run On Nbody systems}
\usepackage{graphicx,natbib,a4wide}
\usepackage[top= 2.25cm, bottom=1.0cm, left=1.55cm, right=1.55cm]{geometry}

\usepackage[T1]{fontenc}
\usepackage[scaled]{helvet}
\renewcommand*\familydefault{\sfdefault} %% Only if the base font of the document is to be sans serif

\newcommand{\mnras}{MNRAS}
\newcommand{\apj}{ApJ}
\newcommand{\nat}{Nature}
\newcommand{\apjl}{ApJL}
\newcommand{\apjs}{ApJS}
\newcommand{\physrep}{Phys.~Rep.}
\newcommand{\aap}{A\&A}
\newcommand{\aaps}{A\&AS}
\newcommand{\araa}{ARA\&A}
\newcommand{\aj}{AJ}
\newcommand{\prd}{PhRvD}
\newcommand{\repprog}{Rep.~Prog.~Phys}


\begin{document}

\title{OBERON - OBliquity and Energy balance Run On Nbody systems}
\author{Duncan Forgan (\texttt{github/dh4gan})}
\maketitle

\newpage
\tableofcontents
\newpage

\section{Getting OBERON}

\noindent OBERON is hosted on github at +++

\section{Compiling OBERON}

\noindent The source code is in the \texttt{/src} folder, which contains a \texttt{Makefile}.  With the appropriate software installed, type 'make' to compile an executable.


\section{Code Design}

\section{Input Options}


\subsection{Global Options}

\begin{itemize}
\item{\texttt{ParType `Positional/Orbital'}} The format of data for entry (either Cartesian co-ordinates for position and velocity or orbital elements)
\item{\texttt{NBodyOutput (string)}} The filename for the N Body data file
\item{\texttt{SnapshotTime (double)}} The time interval between data dumps and snapshots
\item{\texttt{NGridPoints: (integer)}} The number of grid points used by the latitudinal energy balance model
\item{\texttt{MaximumTime: (double)}} The maximum runtime of the simulation
\item{\texttt{SystemName: (string)} A descriptive string for the simulation
\item{\texttt{Number Bodies}: (integer)}} The number of bodies in the system
\item{\texttt{Restart: `T/F'}} is the simulation a restart? True or False (careful with this - not stable currently)
\item{\texttt{FullOutput `T/F'}} True: Output full snapshots as well as log files with surface averaged values for each body, False: log files only
\end{itemize}


\subsection{Body Options (must be specified for each body)}

\begin{itemize}
\item \texttt{BodyName: (string)}
\item \texttt{BodyType: `Star/Planet/World'}
\item{\texttt{Mass: (double)}}
\item{\texttt{Radius: (double)}}
\end{itemize}

For Positional Files:

\begin{itemize}

\item{\texttt{Position: (double) (double) (double)}} Cartesian position vector (astronomical units)
\item{\texttt{Velocity: (double) (double) (double)}} Cartesian velocity vector (2$\pi$ *AU/yr)
\end{itemize}


For Orbital Files:

\begin{itemize}
\item{\texttt{SemiMajorAxis: (double)}} Semimajor Axis (AU)
\item{\texttt{Eccentricity: (double)}} Eccentricity
\item{\texttt{Inclination: (double)}} Inclination (+++units?+++)
\item{\texttt{LongAscend: (double)}} Longitude of the Ascending Node
\item{\texttt{Periapsis: (double)}} Argument of Periapsis
\item{\texttt{MeanAnomaly: (double)}} Mean Anomaly
\item{\texttt{OrbitCentre: (integer)}} Where is the initial orbit focus? -1 = (0,0,0), 0=system centre of mass, 1,2,3... = Body 1,2,3 )
\end{itemize}

\subsubsection{Star Options}

\begin{itemize}
\item{\texttt{Luminosity (double)}} Bolometric Luminosity (solar luminosity)
\end{itemize}

\subsubsection{World Options}

\begin{itemize}
\item{\texttt{RotationPeriod (double)}} World rotation period in days
\item{\texttt{Obliquity (double)}} initial obliquity in degrees
\item{\texttt{WinterSolstice (double)}} orbital longitude of the winter solstice (degrees)
\item{\texttt{OceanFraction (double, $[0.0 \rightarrow 1.0]$)}} Fraction of the world's surface that is ocean
\item{\texttt{InitialTemperature (double)}} Initial surface temperature of the world (at all latitudes)
\item{\texttt{IceMeltingOn `T/F'}} Is latent heat of melting for ice accounted for in climate calculation? True or False 
\end{itemize}

\section{Outputs}:

\begin{itemize}
\item{\texttt{<WorldName>.<number>}} - a  snapshot of \texttt{<WorldName>}'s latitudinal climate properties
\item{\texttt{<WorldName>.<log>}} - a log file for \texttt{<WorldName>} tracking globally averaged climate properties and position/orbital properties
\end{itemize}


An N Body file is also produced (name specified by the user), which is more suited to plotting the entire simulation's evolution (Star and Planet objects included).

%Python scripts for plotting these datafiles can be found in `dh4gan/plot_nbody_EBM`
%
%The code was developed using the eclipse CDT, which auto-generates a Makefile to compile the code.  There is also a manual Makefile in the repository to
%compile with g++.
%

\end{document}
